\documentclass[11pt,twoside,a4paper,titlepage]{report}
\usepackage[a4paper, portrait, margin=1in]{geometry}

\newcommand{\valref}[3]{
	\begin{table}[h!]
	\begin{tabular}{l l}
		\textbf{Symbol} & #1\\
		\textbf{Valid range} & #2\\
		\textbf{Default} & #3
	\end{tabular}
	\end{table} 
}

\begin{document}
\title{CEXT documentation}
\author{J\'{a}n Dupej}
\date{October 16, 2017}

\maketitle
\tableofcontents

\chapter{Introduction}
This tutorial details the operation of CEXT, the cortical area extractor.
The underlying algorithm has been discussed in \cite{dupej2017}.

\chapter{Reference}

\section{Menu}
\subsection{Load batch...}
\subsection{Engage!}
\subsection{Debug selected}

\section{Settings}
Please refer to XXX for specifics on notation.
\subsection{AxisDetection}
This settings specifies the method of identifying the medial axis.
\paragraph{Landmarks} 
Given the landmarks \(\mathbf{p}_0, \mathbf{p}_1\), the medial axis is specified as the straight line between them.
\begin{equation}
\mathbf{a}_i = \mathbf{p}_0 + \big(\mathbf{p}_1 - \mathbf{p}_0\big) \bigg(r_0 + \frac{(i-1)(r_1 - r_0)}{n-1}\bigg);  i=1,..n 
\label{equ:axsDetLandm}
\end{equation}

\paragraph{CrossSectionCentroids}
This setting forces the initial detection of medial axis as in Eq.\ref{equ:axsDetLandm}.
However, these axis points are then refined as follows.
A plane perpendicular to the landmark-defined medial axis is constructed in each \(\mathbf{a}_i\) from the previous step.
Rough segmentation of cortical area is performed by thresholding against \(h_{prior}\), in each reslice.
Continuous elements of the segmented cortical area are determined and all but the largest one (by pixel count) are discarded.
The centroid of the remaining cortical area in the reslice is calculated and stored as \(\mathbf{a}_i\).

\paragraph{CrossSectionLinear}
Cross-section centroids are identified as with the setting \emph{CrossSectionCentroids}.
To these centroids, a straight line \(A: \mathbf{c} + t\mathbf{d}, t\in\mathbf{R}\) is fitted with ordinary least squares.
Points \(\mathbf{c}_0, \mathbf{c}_1 \in A\) are found, such that the slice numbers of these match the slice numbers corresponding to the start and end of medial axis ROI.
The medial axis \(\mathbf{a}_i, i=1,..n\) is then constructed by equidistantly placing points on the straight line between \(\mathbf{c}_0\) and \(\mathbf{c}_1\).

\paragraph{CrossSectionGaussian}
This is the default setting.
A preliminary axis \(\tilde{\mathbf{a}}_i\) is constructed the same as in \emph{CrossSectionCentroids}.
The final medial axis is established by convolving with a Gaussian kernel \(\mathbf{a} = \tilde{\mathbf{a}} \ast k\), where
\begin{equation}
k(x) = exp\bigg(-\frac{x^2}{2\sigma_a^2}\bigg)
\label{equ:axsDetGauss}
\end{equation}
This setting is recommended for most applications.
The medial axis is represented as a general polyline; however, the high-frequency information that can be caused by noise is suppressed.
The width of the kernel \(\sigma_a\), is configured in another field.


\subsection{AxisGaussianBandwidth}
\valref{\(\sigma_a\)}{\(\langle 0, \infty )\)}{3}
This parameter is applicable only when axis detection is set to \emph{CrossSectionGaussian}; with other settings, it is ignored.
It controls the bandwidth of the Gaussian kernel that is used to smoothen the general medial axis.
This value is not given in any length units, but in indices of the medial axis samples.
Larger values will result in a smoother axis, very large values will restrict the responsiveness of the medial axis shape to the shape of the bone.
Low values will cause the medial axis to better follow the shape of the bone; however very low values may not suppress the effect of noise in the data.

\subsection{BoneRoiStart}
\valref{\(r_0\)}{\(\langle 0, 1 \rangle\)}{0.2}
This value specifies the location along the length of the bone, where reslicing starts.

\subsection{BoneRoiEnd}
\valref{\(r_1\)}{\(\langle 0, 1 \rangle\)}{0.8}
This value specifies the location along the length of the bone, where reslicing ends.

\subsection{DebugAll}
Speciefies, if the result of each extraction is shown for diagnostic purposes.
Note that is this is set to \emph{True}, a debugging window will be shown after each extraction and the program proceeds with the next file only after the user closes the debug window.
The result files are generated, regardless how this field is set.

\subsection{MaxBoneRadius}
\valref{}{}{}

\subsection{MedianFilterRadius}
\valref{\(k_{median}\)}{\(\langle 0, 16 \rangle\)}{0}
This value configures the size of the median filter applied to the entire stack before extraction.
Note that the median filtering is performed in 3D.
Higher values will result insmother images, at considerable computational expense.
If set to 0, no filtering is done.

\subsection{NumRays}
\valref{\(n_{ray}\)}{\(\langle 1, 10000\rangle\)}{50}
Sets the number of angle-regular directions along which the cortical boundaries will be sampled.
This is also equal to the number of columns of the cortical thickness matrix that results from the extraction.

\subsection{NumSlices}
\valref{\(n_{slice}\)}{\(\langle 1, 10000\rangle\)}{100}
Sets the number of cross-section reslices that will be produced in the extraction process.
This is also equal to the number of rows of the cortical thickness matrix that results from the extraction.

\subsection{PriorThreshold}
\valref{\(h_{prior}\)}{\(\langle -1000, 3092\rangle\)}{525}
This is the prior threshold separating bone from soft tissues and is given in Hounsfield units. 
This value is used in rough segmentation of cortical area, and in some parts of the extraction, may be automatically refined.

\subsection{RefineEndpoints}
If this is set to \emph{True}, endpoint refinement, as detailed in \cite{dupej2017} will be applied.
It is recommended to enable this only when significant error in endpoint landmarks is expected.
The default value is \emph{False}.

\subsection{ResultPath}
The absolute path, where the resulting files will be saved.
It is not necessary to include the training backslash.

\subsection{UseHmh}
If set to \emph{True}, half-maximum-height adaptive thresholding \cite{spoor1993} will be used.
This will result in sub-pixel resolution in the detection of outer and inner cortical boundaries.
This method is described in more detail in \cite{dupej2017}.
If set to \emph{False}, ordinary thresholding will be used.
The default is \emph{True} and it is recommended to keep this setting.

\section{Batch file format}

\bibliographystyle{plain}
\bibliography{refs}

\end{document}